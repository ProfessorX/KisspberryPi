
%%% Local Variables: 
%%% mode: latex
%%% TeX-master: t
%%% End: 


\documentclass{article}

\begin{document}
The step by step explanation of algorithm 1 in paper 99 are as
follows. 
\begin{enumerate}
\item The core part of paper 99 is their thermal comfort correlation
  model, which emphasizes on the indoor, outdoor temperature and human
  feedback through a mobile app. In short, they see there is a mapping
  between comfort level (by index, from $-3~to~3$).
\item The algorithm begins by setting an initial indoor temperature
  which ``tries'' to minimize the comfort index, in turn they regard
  this as ``maximize'' the human comfort. (In table 1, page 4, the
  other values are ``non-neutral''.)
\item Then they try to check if this temperature satisfies (the voted
  comfort index should be $-1,0,1$, other values are invalid) most
  people (an example threshold is $80\%$).
\item If this temperature makes more than $80\%$ people comfortable,
  then it's chosen.
\item Else, eliminate the person that is the ``pickiest'', a lively
  example would be a person who votes 28 degrees as too cold. (The
  case is that there may be coincide people who vote the same, making
  them the ``pickiest'' at all times. The algorithm should have
  mentioned that in this case a random one should be taken out of
  consideration.) After elimination, restart from step 2 by picking
  another temperature which ``tries to satisfy most people''.
\end{enumerate}
\end{document}
