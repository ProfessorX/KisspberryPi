
%%% Local Variables: 
%%% mode: latex
%%% TeX-master: t
%%% End: 


\documentclass{article}

\begin{document}


\section{Paper77}
\label{sec:paper77}
\begin{itemize}
\item This paper should be categorized into theoretical ones. It
  proclaims a heuristic optimization algorithm used to speed up the
  traditional way of calculating energy consumption. In general, the
  approach this paper takes to induce its algorithms is correct but
  the algorithms are a little self-evident. 
\item Strengths. The authors created a new algorithm which is both
  computationally efficient and more practical than the traditional
  one. 
\item Weaknesses. The main weakness of this paper is in the
  algorithms. There are not some fancy operations but merely patching
  the in-use LP method to their own needs, namely, having a tighter
  restriction. 
\item Generally speaking, this pure theoretical paper is well-written
  with regard to the specific problem it's tackling. There are no
  obvious grammar errors. However, a few citations are missing in the
  paper in section 2.1.2 and 2.1.4. It puts forward two algorithms
  that can be regarded as a mediocre between MILP and LP. These two
  algorithms are created ``in need'', so they are not easy to
  generalize. Moreover, the authors do state that this is a much
  simplified scenario to consider AC energy consumption optimization
  so its validity in real-life is left to be proved. Some updating
  expectations are listed as follows.
  \begin{itemize}
  \item For table 2, you are just using one case data to be tested on
    3 methods, please make the comparison between these 3 more
    versatile, e.g.\ add some different cases so it is easier for
    people to tell their performances. Metrics like standard deviation
    are welcome. The problem is the same when you are comparing those
    methods in case study 4.2. The running time from just one machine
    is not convincing.
  \item There are a great deal of annotations used in this paper,
    please try to rewrite the core part section 2.1 and 2.2 to make
    them easier to understand.
  \end{itemize}
\end{itemize}




 




\section{Paper97}
\label{sec:paper97}

\begin{itemize}
\item This paper is a theoretical one with real-life data collected
  and analyzed by ``best practice'' data mining method,
  i.e. KNN\@. After analyzing zone-based occupancy data, one more
  theoretical HVAC method is put forward to compete against
  traditional ON/OFF one. The simulation of FineTherm method shows a
  over 30\% energy reduction, which will be put into examination in
  real world according to the authors.
\item Strengths. A scale-free metric is created to tackle energy
  efficiency in office buildings. This metric is generalizable so it's
  easy for other researchers to validate. Moreover, the model built to
  reflect occupancy is ``preliminary'' at the same time but reported
  to be accurate. 
\item Weaknesses. Overall, the paper is a good one. The main weakness
  lies in the STEM equation itself, which is also admitted by authors
  in section 2.2. By using this equation we can get a number, but we
  can not infer from it the exact situation in the building, the STEM
  value is high may due to either a high occupancy or a high energy
  consumption.
\item In general, this paper demonstrates its contributions
  properly. We can find few grammatical errors and sufficient
  citations to work done before. From this paper we can get several
  handy equations to use or validate. But the occupancy model may
  suffer in cases like a completely open office, and as with the
  FineTherm approach, more simulations and real world validations are
  expected. Some expected modifications are listed.
  \begin{itemize}
  \item In section 2.2, when you are describing STEM referring to
    ``relevant building code'', please make it more clear, or add the
    specific reference.
  \item Figure 9, 11 and 12 are talking about the same thing. Please
    add annotations for those symbols and reconsider if some of them
    could be removed.
  \item In figure 7 we can see that the variance is significant for
    all 4 subfigures. Please provide any plausible explanation for
    such a high variance.
  \end{itemize}
\end{itemize}




 



\section{Paper99}
\label{sec:paper99}


\begin{itemize}
\item This paper is more or less an experimental one featuring a
  mobile phone application used to do sort of ``real-time'' occupant
  comfort data collection and a pretty straight-forward algorithm to
  dynamically control BMS system, thus improving the overall
  occupants' comfort level.
\item Strengths. Since this paper focuses more on engineering and
  experiments, it is successful with regard to this field. Several
  experiments and simulations are carried out to show that the
  authors' ideas are applicable. 
\item Weaknesses. The TCC model described in section 4 is pretty
  straight-forward and a little preliminary. Before formally building
  their model, the authors make some justifications why they do not
  use the other one which may have a higher accuracy. For this part
  more comparisons are expected.
\item In general, this paper does not overstate its contributions. A
  few experiments were set based on an algorithm to optimize office
  temperature by controlling BMS, with a metric computed using TCC
  model. However, the validity of this TCC model is well worth more
  investigations. For example, what could the results be if we make
  the $L(T_{i},T_{o})$ more complicated? Please consider the following
  modification suggestions.
  \begin{itemize}
  \item Add some comments or rewrite Algorithm 1 in section 6 so that
    it would be easier to understand.
  \item Try to make a better workflow figure to replace figure 8. Or
    you can add more details in section 7.2
  \end{itemize}
\end{itemize}
 












\end{document}
